% !TEX root = ./manual.tex
%%
% Maintainer
% @author Thomas Lehmann
%
\section{Bekannte Probleme}\label{sec:knownissues}\index{Bugs}
Derzeit sind folgende Problem bekannt, zu denen nur Work Arounds existieren.

\subsection{Overleaf: Glossar wird nicht erzeugt }\label{sec:knownissues:overleaf_glossary}
Es wird kein Glossary erzeugt. Hierzu gibt es zwei Workarounds:
\begin{enumerate} 
	\item Die Master-Datei und alle zugehörigen Dateien sollten im Root-Ordner liegen. Somit müssen alle Dateien/Ordner um eine Ebene nach oben verschoben werden, sodass die Master-Datei im Root-Ordner liegt.
    \item Es muss die Make-Datei von Overleaf durch eine eigene Make-Datei im Root-Ordner ergänzt werden. Der Name der Datei ist \texttt{latexmkrc}, siehe \url{https://www.overleaf.com/learn/latex/Articles/How_to_use_latexmkrc_with_Overleaf}. Die Datei muss dann nachfolgende Anweisungen enthalten.
\end{enumerate}
    
\texttt{latexmkrc}:
\footnotesize{
\begin{verbatim} 
add_cus_dep('glo', 'gls', 0, 'makeglo2gls');
sub makeglo2gls {
    system("makeindex -s '$_[0]'.ist -t '$_[0]'.glg -o '$_[0]'.gls '$_[0]'.glo");
}

add_cus_dep('slo', 'sls', 0, 'makeglo2sls');
sub makeglo2sls {
    system("makeindex -s '$_[0]'.ist -t '$_[0]'.slg -o '$_[0]'.sls '$_[0]'.slo");
}

add_cus_dep('acn', 'acr', 0, 'makeacn2acr');
sub makeacn2acr {
    system("makeindex -s \"$_[0].ist\" -t \"$_[0].alg\" -o \"$_[0].acr\" \"$_[0].acn\"");
}
\end{verbatim}
}

Hinweis: Einige PDF-Viewer stellen das senkrechte Hochzeichen als schräges Hochkomma dar, wodurch copy-past nicht funktioniert. Zeichen per suchen-und-ersetzen austauschen.

\subsection{Overleaf: Glossar-Fonts}
Einige Fonts funktionieren nicht. Hier hilft derzeit nur das lokale übersetzen mit einem Standard-LaTeX-Paket.

\subsection{Font-Größe auf dem Deckblatt}
Die Font-Größe auf dem Deckblatt passt sich automatisch an. Dabei wird eine Heuristik eingesetzt. Somit kann es passieren, dass die Font-Größe nicht optimal ist.

Lösung: Die Font-Größe muss direkt in der Datei coverpage.tex gesetzt werden.

\subsection{Lorem Ipsum}
Bei einigen \LaTeX -Distributionen fehlt das Paket lipsum, welches Beispieltexte für das Beispiel liefert. Aufgetreten auf  Ubuntu 18 Systemen.

Lösung: Das Paket aus dem Style-File durch löschen der Zeile herausnehmen. Das Beispiel kann dann nicht verwendet werden. Auf das Template hat es keinen Einfluss.

\subsection{Literaturverzeichnis wird aus IDE nicht erstellt}
Einige IDEs für \LaTeX haben das Bibliothekssystem Biber statt BibTex eingestellt. Dann wird im Compil-Vorgang kein Literaturverzeichnis erzeugt. Compil-Vorgang muss in der IDE angepasst oder in der Shell manuell ausgeführt werden.

\subsection{Font-Warnings}
Ein großteil der Font-Warnings werden durch das Paket \texttt{hyphenat} verursacht und können ignoriert werden. Bei anderen Font-Warnings ist die genaue Ursache noch unbekannt; echte Problem sind derzeit nicht bekannt.